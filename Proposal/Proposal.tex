\documentclass{article}[9pt]
\usepackage{framed}
\usepackage{enumerate}
\usepackage[margin=2.5cm]{geometry}
\begin{document}
\begin{framed}
\noindent
\large{\textbf{CS 4780: Machine Learning \hfill Project Proposal}}
\end{framed}

\noindent\Large{\textbf{The Team}}\\
\noindent We have four members in our team.\\\\
\noindent\Large{\textbf{Motivation}}\\
\noindent Previously, we trained machine learning algorithms on relatively small amounts of filtered training data and performed simple classification on test cases. The project that we have selected, General Electric Company's Kaggle competition, presents us with the opportunity to apply what we have learned to a structured output problem. Our motivation to complete this project comes from our desire to overcome the challenges involved with solving a different type of problem than what we are used to (such as working with sizable amounts of data, some of which may be irrelevant) in order to produce an optimization algorithm that can be used to make future real-life processes more efficient.

The task is also the most complex, on Kaggle, and requires in-depth knowledge of all machine learning techniques. Our motivation to pick this project up, rather than the multitude others was mainly because we would like to challenge ourselves with a complex, real-world machine learning problem that would also help us understand where we fall short.

\\[\baselineskip]
\noindent
\Large{\textbf{The Task}} \\
\noindent The main objective of the solution is to present an optimized flight plan to the pilot, to minimize costs given the training set, which contains information about flight and weather. We are provided with a cost function in a simulator to evaluate the expense undertaken to complete a flight path given predicted weather and ground conditions. True evaluation can only be achieved by submitting to the contest website, in which five submissions are allowed per day.
\begin{itemize}
	\item Since the flight has innumerable variables that can be tweaked to reduce costs, specifying potential suggestions (in other words, the output variables of our solution) is possibly the first design problem that needs to be solved.
	\item Dealing with data larger than system memory
	\item Study the feasibility of distributing the load among several clients to increase the speed of learning/prediction
	\item Generating an optimal path rather than classifying a single instance (structured output)
\end{itemize}
%\\[\baselineskip]
\noindent
\Large{\textbf{General Approach}}%\\
\begin{enumerate}
	\item We will prune the input data to reduce complexity by assigning weights based on the relevance of the features. For example, customer dissatisfaction could be considered much less than weather conditions. 
	\item As the input data is too large to directly work with, we will use methods that don't require loading the whole training set or using ensemble methods. Also, we will transform the data to make it more tractable.
	\item By splitting the data into more manageable chunks, we can construct trained models on subsets of the total data in order to speed up the training process and run our learning algorithm concurrently.
	\item We will have a unifying method for the models to reduce time spent on recomputation and allow for quick modifications on certain data sets or partial retraining.
	\item We will use the model constructed along with the parameters given to determine the most optimal flight path from a source to destination.
	\item We plan to first train for optimal flight speed and altitude given the parameters in the training set which include weight, distance remaining, and the current position of the aircraft. We plan to try various regression techniques like k-nearest neighbor and SVM for regression. With these parameters, we plan to follow the shortest path which is along the great circle. Further enhancements include adapting to geography (jetstreams may make a divergent path cost less), weather, or turbulent zones.
\end{enumerate}
%\\[\baselineskip]
\noindent
\Large{\textbf{Resources}}
\begin{itemize}
	\item Data sets provided to us by the GE Kaggle competition site
	\item Software: 
		\begin{enumerate}[(a)]
			\item FlightQuest Simulator (FQS) provided to us by the GE Kaggle competition site to produce regression values
			\item Microsoft Visual Studio 2012 to used for read the FQS source code
			\item Python 2.7 (and libraries) to develop our data processing and machine learning algorithms
			\item SVMperf to perform SVM regression
		\end{enumerate}
	\item Readings on structured output. 
\end{itemize}
\hspace*{\fill}\\
\Large{\textbf{Schedule}}
\noindent
\begin{itemize}
	\item Nov. 1: Complete research and have understanding of tools and design a basic approach to handling huge dataset.
	\item Nov. 8: Have a working method for generating paths while ignoring most factors
	\item Nov. 22: Have methods implemented to take into account different factors
	\item Nov. 29: Begin final report and poster
	\item Dec 7: Finish refinements on code and report 
\end{itemize}
\end{document}